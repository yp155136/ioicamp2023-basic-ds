\documentclass[standalone]{beamer}

\begin{document}
\section{並查集}

\begin{frame}{並查集}
  \begin{itemize}
    \item 支援合併及查詢集合的資料結構
    \item 具體來說,需要支援:
    \begin{enumerate}
      \item init($n$) : 初始 $n$ 個元素,各自屬於一個獨立的集合。
      \item union($a, b$) : 將 $a$ 元素所在的集合以及 $b$ 元素所在的集合合併。
      \item find($a$) : 詢問 $a$ 所在的集合編號。
    \end{enumerate}
    \item 維護的集合兩兩互斥
    \item 來畫圖解釋吧!
  \end{itemize}
\end{frame}

\begin{frame}{\btitle{實作方法}}
  \begin{itemize}
    \item 將每個集合以樹狀結構來表示
    \item 整個並查集會形成一個森林,而每棵樹分別代表一個集合
    \item 每個元素記錄自身的父節點(程式碼中的 $p$),而每棵樹便以根節點($p[i] = i$ 的點)作爲代表
    \item 在做集合相關操作的時候,\textbf{必須要用代表點(也就是根節點)}來操作
  \end{itemize}
\end{frame}

\begin{frame}[fragile]{\btitle{範例實作}}
  \begin{minted}[breaklines]{cpp}
    struct DisjointSet {
      int p[maxn];
      void init(int n) {
        for (int i = 1; i <= n; i++)
          p[i] = i;
      }
      int find(int x) { return x == p[x] ? x : find(p[x]); }
      void join(int x, int y) { p[find(x)] = find(y); }
    };
  \end{minted}
\end{frame}

\begin{frame}{\btitle{時間複雜度}}
  \begin{itemize}
    \item \texttt{init()}:就是個 $O(N)$
    \item \texttt{find(x)}:看起來好像很快,但其實 worst case 會到 $O(N)$(樹是一條鏈的狀況)
    \item \texttt{join(x, y)}:就是個 $O(1)$
    \item 想想看,\texttt{find(x)} 真的需要 $O(N)$ 那麼多嗎?
  \end{itemize}
\end{frame}

\begin{frame}[fragile]{\btitle{範例實作}}
  \begin{itemize}
    \item \texttt{find()} 事實上就是在樹上不斷往父節點爬
    \item 把查詢路徑上所有人的父節點設爲根節點!
    \item 關心的只是每個節點屬於哪個集合,樹狀結構長什麼樣子並不重要
    \item \texttt{return x == p[x] ? x : p[x] = find(p[x]);}
    \item 經過這個改善後,複雜度\textbf{均攤}下來是 $O(N \log N)$
    \item \textbf{路徑壓縮}
  \end{itemize}
\end{frame}

\begin{frame}[fragile]{\btitle{範例實作}}
  \begin{itemize}
    \item 那 \texttt{join()} 呢?有沒有比較厲害的 join 方式可以降低複雜度?
    \item \textbf{union by rank}
    \item 維護每棵樹當前的深度
    \item 每次要合併時,都將深度小的接到深度大的上
    \item 於是,這樣樹的深度就不會超過 $O(\log n)$,\texttt{find()} 在沒有路徑壓縮下複雜度也是好的
  \end{itemize}
\end{frame}

\begin{frame}[fragile]{\btitle{範例實作}}
  \begin{minted}[breaklines]{cpp}
    void join(int x, int y) {
      x = find(x), y = find(y);
      if (x == y)
        return;
      if (rk[x] < rk[y]) // union by rank
        swap(x, y);
      p[y] = x;
      if (rk[x] == rk[y])
        ++rk[x];
    }
  \end{minted}
\end{frame}

\begin{frame}[fragile]{\btitle{實作總結}}
  \begin{itemize}
    \item 什麼都不做會退化到 $O(N)$
    \item union by rank / 路徑壓縮只用一個的話可以讓複雜度變 $O(\log N)$
    \item 同場加碼:union by rank 跟路徑壓縮一起做的話,均攤時間複雜度爲 $O( \alpha( n ) )$,幾乎可以視為 $O(1)$!
    \item 同場加碼:union by size (小樹接到大樹)跟 union by rank 的效果是一樣的喔!
  \end{itemize}
\end{frame}

\begin{frame}{\btitle{例題}}
  \begin{problem}[經典問題(這題講義上沒有)]
    給你一張 $N$ 個點的圖 $G$,圖一開始沒有任何邊,以及 $M$ 筆操作。

    操作有三種:
    \begin{itemize}
      \item 加邊:給 $a, b$,把點 $a$ 跟點 $b$ 中間加上一條邊
      \item 詢問一:給 $a, b$,問 $a, b$ 是否在同一個連通塊
      \item 詢問二:詢問目前的圖有多少個連通塊
    \end{itemize}

    \begin{itemize}
      \item $N, M \leq 10^5$
    \end{itemize}
  \end{problem}
\end{frame}

\begin{frame}{並查集}
  \begin{itemize}
    \item 並查集很常用的地方:維護圖的連通性
    \item 並查集的集合\textbf{維護連通塊}
    \item 一開始圖是空的,相當於所有點都在自己的連通塊裡面,對應到\texttt{init(N)}!
    \item 有一條邊新增進來的時候,相當於把兩個連通塊變成一個連通塊
    \item 也相當於是並查集的 \texttt{join()} 操作!
  \end{itemize}
\end{frame}

\begin{frame}{並查集}
  \begin{itemize}
    \item 詢問兩點是否連通?
    \item 判斷兩個點是不是在同一個集合
    \item \texttt{djs.find(a) == djs.find(b)}
    \item 詢問連通塊個數?
    \item 一邊 \texttt{join()} 一邊維護
    \item 每當 \texttt{join()} 兩個不同的集合時,連通塊個數\texttt{ -= 1}。
  \end{itemize}
\end{frame}

\begin{frame}{\btitle{例題}}
  \begin{problem}[Ural 1671 Anansis Cobweb]
    給一張無向圖 $G$,$N$ 個點、$M$ 條邊。
    $Q$ 筆操作,每筆操作破壞一條邊後問連通塊個數。

    \begin{itemize}
        \item
            $1 \leq N \leq 10^5$
        \item
            $1 \leq M \leq 10^5$
        \item
            $1 \leq Q \leq M$
    \end{itemize}
  \end{problem}
\end{frame}

\begin{frame}{並查集}
  \begin{itemize}
    \item 但,這題是破壞邊,不是增加邊,怎麼辦?
  \end{itemize}
\end{frame}

\begin{frame}{並查集}
  \begin{itemize}
    \item \textbf{時間倒流}
    \item 沒有被破壞的邊表示從頭到尾都沒被破壞,可以在此刻維護一個並查集
    \item 破壞一條邊,就相當於建造一條邊!
  \end{itemize}
\end{frame}

\begin{frame}{\btitle{例題}}
  \begin{problem}[POJ 1182 食物鏈]
    有三類動物 $A, B, C$,這三類動物的食物鏈構成如下:$A$ 吃 $B$, $B$ 吃 $C$,$C$ 吃 $A$。 
    現有 $N$ 個動物,編號 $1, 2, \ldots, N$。
    每個動物都是 $A,B,C$ 中的一種,但並不知道是哪一種。 
    依序給 $K$ 條屬於以下兩種的敘述:
    \begin{itemize}
      \item \texttt{1 X Y},表示 $X$ 和 $Y$ 是同類
      \item \texttt{2 X Y},表示 $X$ 吃 $Y$
    \end{itemize}
    然而,並不是每條描述都是正確的,有些是真話,有些是假話。
    
    如果當前的話與前面的某些真話衝突,就是假話
    
    請判斷哪些話是真話,哪些話是假話。
  \end{problem}
\end{frame}

\begin{frame}{並查集}
  \begin{itemize}
    \item 並查集的經典變化!
    \item 第一種資訊,顯然可以用並查集維護
    \item 假設 $a, b$ 同類,可能可以類似 \texttt{djs.join(a, b)}
    \item 但,第二種資訊怎麼辦?好像沒辦法直接維護ㄋㄟ
  \end{itemize}
\end{frame}

\begin{frame}{並查集}
  \begin{itemize}
    \item 那就直接假設每種動物在每個分類底下的狀況!
    \item 把每個動物 $i$ 分成三個資訊:動物 $i$ 屬於第 $A$ 類、動物 $i$ 屬於第 $B$ 類、動物 $i$ 屬於第 $C$ 類
    \item 假設 $i_O$ 代表動物 $i$ 是第 $O$ 類的資訊
    \item 有了這個可以做什麼 OuO
  \end{itemize}
\end{frame}

\begin{frame}{並查集}
  \begin{itemize}
    \item 我們可以用並查集維護\textbf{資訊間的因果關係}
    \item 一個集合裡面的資訊代表「這些資訊必須同時發生」
    \item 比如說,假設 $(x, y)$ 符合第一種資訊
    \item 那就代表:如果 $x$ 是 $A$ 類,那 $y$ 必須是 $A$ 類,反之亦然($B, C$ 類同樣可以套用)
    \item 翻譯一下:如果 $x_A$ 發生的話,那 $y_A$ 也一定要發生($B, C$ 類同樣可以套用)
    \item $join(x_A, y_A), join(x_B, y_B), join(x_C, y_C)$
  \end{itemize}
\end{frame}

\begin{frame}{並查集}
  \begin{itemize}
    \item 我們可以用並查集維護\textbf{資訊間的因果關係}
    \item 一個集合裡面的資訊代表「這些資訊必須同時發生」
    \item 比如說,假設 $(x, y)$ 符合第二種資訊
    \item 那就代表:
    \begin{itemize}
      \item 如果 $x$ 是 $A$ 類,那 $y$ 必須是 $B$ 類,反之亦然
      \item 如果 $x$ 是 $B$ 類,那 $y$ 必須是 $C$ 類,反之亦然
      \item 如果 $x$ 是 $C$ 類,那 $y$ 必須是 $A$ 類,反之亦然
    \end{itemize}
    \item $join(x_A, y_B), join(x_B, y_C), join(x_C, y_A)$
  \end{itemize}
\end{frame}

\begin{frame}{並查集}
  \begin{itemize}
    \item 我們可以用並查集維護\textbf{資訊間的因果關係}
    \item 一個集合裡面的資訊代表「這些資訊必須同時發生」
    \item 判斷真假?
    \item 一種判斷方法:如果 join 後發生 $find(x_A) == find(x_B)$,就代表這中間一定有假話!
  \end{itemize}
\end{frame}

\begin{frame}{\btitle{例題}}
  \begin{problem}[經典問題]
    給一張無向圖 $G$,$N$ 個點、$M$ 條邊。
    
    (版本一)請你判斷這張圖是不是二分圖?

    (版本二)請你判斷這張圖存不存在長度是奇數的簡單環?
  \end{problem}

  這題是上個例題的簡化版,也是一個並查集非常實用的地方ㄛ!
\end{frame}

\begin{frame}{\btitle{總結}}
  \begin{itemize}
    \item 並查集是一個維護集合的工具
    \item 常用在圖論上維護圖的連通性
    \item 有蠻酷的變形
    \item 有時候會跟時間線段樹搭配,歡迎參考進階資料結構的講義!
  \end{itemize}
\end{frame}

\end{document}
