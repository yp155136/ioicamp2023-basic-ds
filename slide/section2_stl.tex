\documentclass[standalone]{beamer}

\begin{document}
\section{好用的 STL}

\begin{frame}{好用的 STL}
  \begin{itemize}
    \item STL 裡面有許多實用的東西
    \item 這個章節會選一些介紹,剩下的就麻煩大家自行上網搜尋ㄛ
  \end{itemize}
\end{frame}

\begin{frame}{\btitle{bitset 例題}}
  \begin{problem}[NEOJ 743 數字總和]
    給你 $N$ 個數字 $a_1, a_2, \dots, a_N$,並且滿足 $\sum_{i = 1}^{N} a_i \le M$。

    現在,對於每個數字 $x$ ,請判斷 $x$ 可不可由 $a_1, a_2, \dots, a_N$ 給湊出來。也就是說,能不能找到一個集合 $S$ ,使得 $\sum_{i \in S} a_i = x$。

    \begin{itemize}
        \item $N \leq 2 \times 10^5$
        \item $M \leq 2 \times 10^5$
    \end{itemize}
  \end{problem}
\end{frame}

\begin{frame}[fragile]{\btitle{bitset}}
  \begin{itemize}
    \item 我會 DP!
    \item
    \begin{minted}[breaklines]{cpp}
      dp[0] = 1;
      for (int i = 1; i <= n; ++i) {
        std::cin >> a[i];
        for (int x = m; x >= a[i]; --x) {
          dp[x] |= dp[x - a[i]];
        }
      }
    \end{minted}
    \item 複雜度是 $O(NM)$,TLE
    \item bitset 該出場了! 
  \end{itemize}
\end{frame}

\begin{frame}{\btitle{bitset}}
  \begin{itemize}
    \item 可以想像成一個大 bool 陣列
    \item 可以直接對這個 bool 陣列做位元操作(and, or, 左移等等)
    \begin{itemize}
      \item 做這些操作的複雜度是 $O(\frac{N}{w})$!
      \item $w$ 通常不是 $32$ 就是 $64$ ,後面會解釋!
    \end{itemize}
    \item 可是等等,上面那個轉移式可以被寫成兩個 bool 陣列做位元操作的形式嗎?
  \end{itemize}
\end{frame}

\begin{frame}[fragile]{\btitle{範例實作}}
  \begin{itemize}
    \item
    \begin{minted}[breaklines]{cpp}
      bool dp_tmp[kN] = {false};
      for (int x = a[i]; x <= m; ++x) { // step 1
        dp_tmp[x] = dp[x - a[i]];
      }
      for (int x = 0; x <= m; ++ x) { // step 2
        dp[x] = dp[x] | dp_tmp[x];
      }
    \end{minted}
    \item step 1 其實就是把 $dp$ 整個陣列往右移 $a[i]$ 格
    \item step 2 其實就是 $dp$ 跟 $dp\_tmp$ 做 and 操作
  \end{itemize}
\end{frame}

\begin{frame}[fragile]{\btitle{範例實作}}
  \begin{minted}[breaklines]{cpp}
    std::bitset<kN> dp; // 宣告一個大小為 kN 的 bitset
    dp[0] = 1; // dp[0] 可以直接存取 bitset 的第 0 個 bit,也可以直接寫值
    for (int i = 1; i <= n; ++i) {
      std::cin >> a[i];
      dp |= (dp << a[i]); // dp = dp | (dp << a[i]);
    }
  \end{minted}
\end{frame}

\begin{frame}{\btitle{bitset}}
  \begin{itemize}
    \item 複雜度瞬間變成 $O(\frac{NM}{w})$,AC
    \item $w$ 要看跑程式機器的 word 大小,通常不是 $32$ 就是 $64$
    \item 官解複雜度有除 32 或 64 通常代表有用到 bitset
    \item bitset 的更多用法歡迎參考講義喔!
  \end{itemize}
\end{frame}

\begin{frame}{\btitle{離散化}}
  \begin{itemize}
    \item 詳細的定義可以參考講義
    \item 白話文:把數字重新從 $1$ 開始按照大小填
    \item $[5, -6, 3, 5, 7]$ 變成 $[3, 1, 2, 3, 4]$
    \item 通常離散化後的數字都會是整數
    \item 常見用途:把 $[1, 10^9]$ 的數字變成 $[1, N]$ 後,就可以開一個大小是 $O(N)$ 的陣列做事
  \end{itemize}
\end{frame}

\begin{frame}[fragile]{\btitle{idea 1}}
  \begin{itemize}
    \item 把那 $N$ 個數字丟進 \texttt{std::map} 裡面
    \item 再依序把數字在 map 中對應到的 value 依序填上 $[1, N]$
    \item 最後利用 map 把原本的數字替換成新的數字
    \begin{minted}[breaklines]{cpp}
      std::map<int, int> mp;
      for (int i : a) mp[i] = 0;
      int idx = 0;
      for (auto p : mp) mp[p.first] = ++idx;
      for (int &i : a) i = mp[i];
    \end{minted}
    \item \texttt{std::map} 常數太大了,實作上並不是一個有效率的作法
  \end{itemize}
\end{frame}

\begin{frame}[fragile]{\btitle{idea 1}}
  \begin{itemize}
    \item 把所有數字由小到大排序好
    \item 利用二分搜尋法,來找出每個數字是第幾小
    \item 比如說,$[5, -6, 3, 5, 7]$ 的 $5$ 是第三小,就把 $5$ 對應到 $3$
  \end{itemize}
\end{frame}

\begin{frame}[fragile]{\btitle{範例實作}}
  \begin{minted}[breaklines]{cpp}
    std::vector<int> v;
    for (int i = 1; i <= n; ++i) {
      v.emplace_back(arr[i]);
    }
    std::sort(v.begin(), v.end());
    for (int i = 1; i <= n; ++i) {
      arr[i] = std::lower_bound(v.begin(), v.end(), arr[i]) - v.begin() + 1;
    }
  \end{minted}
  
  \texttt{std::sort} 可以用 $O(N \log N)$ 的時間來排序那 $N$ 個數字

  \texttt{std::lower\_bound} 可以在 $O(\log N)$ 的時間內找出第一個不小於目標數字的位置
\end{frame}

\begin{frame}[fragile]{\btitle{範例實作}}
  \begin{itemize}
    \item 
    \begin{minted}[breaklines]{cpp}
      std::vector<int> v;
      for (int i = 1; i <= n; ++i) {
        v.emplace_back(arr[i]);
      }
      std::sort(v.begin(), v.end());
      v.resize(std::unique(v.begin(), v.end()) - v.begin());
      for (int i = 1; i <= n; ++i) {
        arr[i] = std::lower_bound(v.begin(), v.end(), arr[i]) - v.begin() + 1;
      }
    \end{minted}

    \item \texttt{std::unique} 會把所有不同的元素搬到容器的前面,並且回傳最後一個不同元素後面的位置
    \item 拿這個回傳值跟容器的開頭相減,就可以得到總共有多少相異的元素(resize 那行在做的事情)
  \end{itemize}
\end{frame}

\begin{frame}{\btitle{離散化}}
  \begin{itemize}
    \item \texttt{std::sort} 可以用 $O(N \log N)$ 的時間來排序那 $N$ 個數字
    \item \texttt{std::lower\_bound} 可以在 $O(\log N)$ 的時間內找出第一個不小於目標數字的位置
    \item \texttt{std::unique} 會把所有不同的元素搬到容器的前面,並且回傳最後一個不同元素後面的位置,複雜度為 $O(N)$
  \end{itemize}
\end{frame}

\begin{frame}{\btitle{pbds}}
  \begin{itemize}
    \item 平板電視、黑魔法
    \item 有現成的 hash table、heap、平衡樹
    \item 平衡樹比 \texttt{std::set} 還強大,可以支援搜尋一個數字的 order (大小)
    \item 但其實不會 pbds 對競程影響不會那麼大(?)
    \item 有興趣請參考講義 + 網路資料!
  \end{itemize}
\end{frame}

\begin{frame}{\btitle{rope}}
  \begin{itemize}
    \item 簡化版持久化線段樹
    \item 有興趣請參考講義 + 網路資料!
  \end{itemize}
\end{frame}

\end{document}



