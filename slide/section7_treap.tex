\documentclass[standalone]{beamer}

\begin{document}
\section{樹堆}

\begin{frame}[fragile]{樹堆}
  \begin{itemize}
    \item Treap
    \item Tree + Heap
    \item 樹堆是一種二元樹,是 Binary Search Tree + Heap 的綜合題
    \item 節點有兩個值:pri 跟 key,分別滿足\textbf{二元搜尋樹}的性質跟\textbf{堆}的性質
  \end{itemize}
\end{frame}

\begin{frame}[fragile]{樹堆}
  \begin{itemize}
    \item 樹性質:
    \begin{itemize}
      \item 左子樹的 key 都小於根節點的 key
      \item 左子樹的 key 都大於根節點的 key
      \item 左右子樹都滿足樹性質
    \end{itemize}
    \item 堆性質:
      \begin{itemize}
        \item 左子樹的 pri 都小於跟節點的 pri
        \item 左子樹的 pri 都小於跟節點的 pri
        \item 左右子樹都滿足堆性質
      \end{itemize}
    \item pri 是用\textbf{隨機生成}的,可以讓樹高期望上變成 $O(\log N)$
  \end{itemize}
\end{frame}

\begin{frame}[fragile]{樹堆}
  \begin{itemize}
    \item 兩個最重要的操作:
    \begin{itemize}
      \item $merge(a, b)$ : 將兩棵樹堆 $a, b$ 合併爲一棵樹堆,需滿足樹堆 $a$ 的所有 key 值均小於等於樹堆 $b$ 的所有 key 值
      \item $split(t, k)$ : 將樹堆 $t$ 分爲兩棵樹堆,其中第一棵樹堆中的 key 值均小於等於 $k$,而第二棵樹堆中的 key 值均大於 $k$
    \end{itemize}
  \end{itemize}
\end{frame}

\begin{frame}[fragile]{樹堆}
  \begin{minted}[breaklines]{cpp}
    struct Treap {
      Treap *l, *r;
      int pri, key, val;
      Treap(int _val, int _key)
          : val(_val), key(_key), l(NULL), r(NULL), pri(rand()) {}
    };
  \end{minted}
\end{frame}

\begin{frame}[fragile]{\btitle{merge}}
  \begin{itemize}
    \item 整個合併過程因爲堆性質的引入而拯救世界。由於\textbf{已經保證樹堆 $a$ 的所有 key 值需小於等於樹堆 $b$ 的所有 key 值},我們只要顧好堆性質即可!因此,整個合併過程可以簡述如下:
    \item 
      合併三部曲:
      \begin{enumerate}
        \item 其中一棵樹爲空,直接返回另一棵
        \item 依照堆性值決定哪個節點當根
        \item 合併尚未合併的那一邊子樹
      \end{enumerate}
  \end{itemize}
\end{frame}

\begin{frame}[fragile]{\btitle{merge}}
  \begin{itemize}
    \item 
      合併三部曲:
      \begin{enumerate}
        \item 其中一棵樹爲空,直接返回另一棵
        \item 依照堆性值決定哪個節點當根
        \item 合併尚未合併的那一邊子樹
      \end{enumerate}
  \end{itemize}
\end{frame}

\begin{frame}[fragile]{\btitle{merge}}
  \begin{minted}[breaklines]{cpp}
    Treap *merge(Treap *a, Treap *b) {
      if (!a || !b) return a ? a : b;
      if (a->pri > b->pri) {
        a->r = merge(a->r, b);
        return a;
      } else {
        b->l = merge(a, b->l);
        return b;
      }
    }
  \end{minted}
\end{frame}

\begin{frame}[fragile]{\btitle{split}}
  \begin{itemize}
    \item 在分裂的過程中,由於已經\textbf{維護好堆性質的部分},因此,只需要依 key 值來決定要分到哪一棵樹,整個分裂過程簡述如下:

    \item 
    分裂三部曲:
    \begin{enumerate}
      \item 樹爲空,直接返回
      \item 根節點要送給誰(依 key 值決定)
      \item 分裂尚未分裂的那一棵子樹
    \end{enumerate}
  \end{itemize}
\end{frame}

\begin{frame}[fragile]{\btitle{split}}
  \begin{itemize}
    \item 
    分裂三部曲:
    \begin{enumerate}
      \item 樹爲空,直接返回
      \item 根節點要送給誰(依 key 值決定)
      \item 分裂尚未分裂的那一棵子樹
    \end{enumerate}
  \end{itemize}
\end{frame}

\begin{frame}[fragile]{\btitle{split}}
  \begin{minted}[breaklines]{cpp}
    void split(Treap *t, int k, Treap *&a, Treap *&b) {
      if (!t) a = b = NULL;
      else if (t->key <= k) {
        a = t;
        split(t->r, k, a->r, b);
      } else {
        b = t;
        split(t->l, k, a, b->l);
      }
    }
  \end{minted}
\end{frame}

\begin{frame}[fragile]{\btitle{insert, remove}}
  \begin{itemize}
    \item insert
    \begin{minted}[breaklines]{cpp}
      Treap *insert(Treap *t, int k) {
        Treap *tl, *tr;
        split(t, k, tl, tr);
        return merge(tl, merge(new Treap(k), tr));
      }
    \end{minted}
    \item remove
    \begin{minted}[breaklines]{cpp}
      Treap *remove(Treap *t, int k) {
        Treap *tl, *tr;
        split(t, k - 1, tl, t);
        split(t, k, t, tr);
        // return merge(tl, tr);
        return merge(merge(tl, t->l), merge(t->r, tr));
      }
    \end{minted}
  \end{itemize}
\end{frame}

\begin{frame}[fragile]{\btitle{例題}}
  \begin{itemize}
    \item 要怎麼轉換到序列上呢?
    \item 把 key 當作序列的 index,並且多一些變數維護題目要求的數值
    \item 直接來看題目!
    \begin{problem}[區間最大值 Range Maximum Query,RMQ]
      一開始給你 $n$ 個數字 $a_1, a_2, \ldots, a_n$,
      接下來有 $m$ 個操作,操作有兩種:
      \begin{enumerate}
          \item 將 $a_p$ 改為 $x$。
          \item 詢問 $a_l, a_{l+1}, \ldots, a_r$ 中的最大值。
      \end{enumerate}
      假設所有數字均為正整數
    \end{problem}
  \end{itemize}
\end{frame}

\begin{frame}[fragile]{\btitle{例題實作}}
  \begin{minted}[breaklines]{cpp}
    struct Treap {
      Treap *l, *r;
      int pri, key; // key:陣列 index
      int val, mx; // val: 該 index 的數值,mx:該子樹的最大值
      Treap() {}
      Treap(int _key, int _val) : l(NULL), r(NULL), pri(rand()), key(_key), val(_val), mx(_val){}
    };
    inline int mx(Treap *t) { return t ? t->mx : 0; }
    inline void pull(Treap *t) { t->mx = max(mx(t->l), max(t->val, mx(t->r))); }
  \end{minted}
\end{frame}

\begin{frame}[fragile]{\btitle{例題實作}}
  \begin{minted}[breaklines]{cpp}
    Treap *merge(Treap *a, Treap *b) {
      if (!a || !b)
        return a ? a : b;
      if (a->pri > b->pri) {
        a->r = merge(a->r, b);
        pull(a); // 子樹的值有動過,更新父節點!
        return a;
      } else {
        b->l = merge(a, b->l);
        pull(b);
        return b;
      }
    }
  \end{minted}
\end{frame}

\begin{frame}[fragile]{\btitle{例題實作}}
  \begin{minted}[breaklines]{cpp}
    void split(Treap *t, int x, Treap *&a, Treap *&b) {
      if (!t)
        a = b = NULL;
      else if (t->key <= x) {
        a = t;
        split(t->r, x, a->r, b);
        pull(a); // 子樹的值有動過,更新父節點!
      } else {
        b = t;
        split(t->l, x, a, b->l);
        pull(b);
      }
    }
  \end{minted}
\end{frame}

\begin{frame}[fragile]{\btitle{例題實作}}
  \begin{minted}[breaklines]{cpp}
    Treap *t = NULL;
    for (int i = 1; i <= n; i++) {
      t = merge(t, new Treap(i, a[i]));
    }
  \end{minted}
\end{frame}

\begin{frame}[fragile]{\btitle{例題實作}}
  \begin{itemize}
    \item 查詢最大值
      \begin{minted}[breaklines]{cpp}
        split(t, l - 1, tl, t);
        split(t, r, t, tr);
        printf("%d\n", mx(t));
        t = merge(merge(tl, t), tr);
      \end{minted}
    \item 修改
      \begin{minted}[breaklines]{cpp}
        split(t, p - 1, tl, t);
        split(t, p, t, tr);
        t->val = t->mx = x;
        t = merge(merge(tl, t), tr);
      \end{minted}
  \end{itemize}
\end{frame}

\begin{frame}[fragile]{\btitle{例題}}
  \begin{itemize}
    \item 還可以加懶標!
    \begin{problem}[區間最大值 Range Maximum Query,RMQ]
      給定一個長度為 \(N\) 的正整數序列 \(a_1, \dots, a_N\),接著有 \(Q\) 個詢問,每個詢問形如
      \begin{itemize}
          \item
              \(\texttt{1 l r}\),請輸出 \(a_l, a_{l+1}, \dots, a_r\) 當中最大的數字是多少。
          \item
              \(\texttt{2 l r d}\),讓 \(a_l, a_{l+1}, \dots, a_r\) 全部增加 \(d\)。
      \end{itemize}

      \begin{itemize}
          \item \(N, Q \leq 5 \times 10^5\)
          \item \(d > 0\)
      \end{itemize}
    \end{problem}
  \end{itemize}
\end{frame}

\begin{frame}[fragile]{\btitle{例題實作}}
  \begin{minted}[breaklines]{cpp}
    struct Treap {
      Treap *l, *r;
      int pri, key, val, big, add;
      Treap(int _key, int _val): l(NULL), r(NULL), pri(rand()), key(_key), val(_val), big(_val), add(0){}
    };
    int Big(Treap *t) {
      return t ? t->big : -inf;
    }
    void pull(Treap *t) {
      t->big = max(t->val, max(Big(t->l), Big(t->r)));
    }
  \end{minted}
\end{frame}

\begin{frame}[fragile]{\btitle{例題實作}}
  \begin{minted}[breaklines]{cpp}
    void push(Treap *t) {
      if (t->add == 0)
        return;
      if (t->l) {
        t->l->val += t->add;
        t->l->big += t->add;
        t->l->add += t->add;
      }
      if (t->r) {
        t->r->val += t->add;
        t->r->big += t->add;
        t->r->add += t->add;
      }
      t->add = 0;
    }
  \end{minted}
\end{frame}

\begin{frame}[fragile]{\btitle{例題實作}}
  \begin{minted}[breaklines]{cpp}
    Treap *merge(Treap *a, Treap *b) {
      if (!a || !b)
        return a ? a : b;
      if (a->pri > b->pri) {
        push(a); // 要前往子樹了,把懶標推下去
        a->r = merge(a->r, b);
        pull(a); // 子樹的值有動過,更新父節點!
        return a;
      } else {
        push(b);
        b->l = merge(a, b->l);
        pull(b);
        return b;
      }
    }
  \end{minted}
\end{frame}

\begin{frame}[fragile]{\btitle{例題實作}}
  \begin{minted}[breaklines]{cpp}
    void split(Treap *t, int x, Treap *&a, Treap *&b) {
      if (!t)
        a = b = NULL;
      else if (t->key <= x) {
        a = t;
        push(a); // 要前往子樹了,把懶標推下去
        split(t->r, x, a->r, b);
        pull(a); // 子樹的值有動過,更新父節點!
      } else {
        b = t;
        push(b);
        split(t->l, x, a, b->l);
        pull(b);
      }
    }
  \end{minted}
\end{frame}

\begin{frame}[fragile]{\btitle{例題實作}}
  \begin{minted}[breaklines]{cpp}
    int query(int l, int r) {
      Treap *tl, *tr;
      split(root, l - 1, tl, root);
      split(root, r, root, tr);
      int res = Big(root);
      root = merge(merge(tl, root), tr);
      return res;
    }
  \end{minted}
\end{frame}

\begin{frame}[fragile]{\btitle{例題實作}}
  \begin{minted}[breaklines]{cpp}
    void add(int l, int r, int d) {
      Treap *tl, *tr;
      split(root, l - 1, tl, root);
      split(root, r, root, tr);
      root->val += d;
      root->big += d;
      root->add += d;
      root = merge(merge(tl, root), tr);
    }
  \end{minted}
\end{frame}

\begin{frame}[fragile]{\btitle{維護 size}}
  \begin{itemize}
    \item 維護一個節點底下有多少個節點!
    \begin{minted}[breaklines]{cpp}
      struct Treap {
        Treap *l, *r;
        int pri, size, val;
        Treap(int _val) : val(_val), size(1), l(NULL), r(NULL), pri(rand()) {}
      };
      int Size(Treap *t) { return t ? t->size : 0; }
      void pull(Treap *t) { t->size = 1 + Size(t->l) + Size(t->r); }
    \end{minted}
  \end{itemize}
\end{frame}

\begin{frame}[fragile]{\btitle{維護 size}}
  \begin{itemize}
    \item split 就多了一種方法:現在我們可以支援把前 $k$ 個東西切下來!
    \begin{minted}[breaklines]{cpp}
      void split(Treap *t, int k, Treap *&a, Treap *&b) {
        if (!t)
          a = b = NULL;
        else if (Size(t->l) + 1 <= k) {
          a = t;
          split(t->r, k - Size(t->l) - 1, a->r, b);
          pull(a);
        } else {
          b = t;
          split(t->l, k, a, b->l);
          pull(b);
        }
      }
    \end{minted}
  \end{itemize}
\end{frame}

\begin{frame}[fragile]{\btitle{維護 size}}
  \begin{itemize}
    \item 有了 size,在一般的序列我們就可以直接把 key 丟掉了!
    \item 其實 key 根本就只是\textbf{它在整個樹堆裡有幾個人比他小嘛!}
    \item \textbf{反正就算沒有 key,這棵樹總是有個中序走訪,我們依然可以將樹對應到序列上}
    \item 放下 key 人生就會豁然開朗
    \begin{problem}[區間最大值 Range Maximum Query,RMQ]
      給一個長度為 $N$ 的序列 $a_1, a_2, \dots, a_N$ 以及 $Q$ 筆操作,操作內容形式如下:
      
      \begin{itemize}
        \item \texttt{1 l r}:把 $a_l$ 到 $a_r$ 區間反轉
        \item \texttt{2 l r}:求出 $a_l$ 到 $a_r$ 的和
      \end{itemize}
    \end{problem}
  \end{itemize}
\end{frame}

\begin{frame}[fragile]{\btitle{例題講解}}
  \begin{itemize}
    \item 抱著 key 不放,會讓序列 index 難以維護
    \item 用 size 讓人生豁然開朗!
    \begin{minted}[breaklines]{cpp}
      struct Treap {
        Treap *l, *r;
        ll pri, size, sum, val;
        bool tag; // 用來維護子樹是否需要翻轉(swap)
        Treap(int _val) : l(NULL), r(NULL), pri(rand()), size(1), sum(_val), val(_val), tag(false) {}
      };
    \end{minted}
  \end{itemize}
\end{frame}

\begin{frame}[fragile]{\btitle{例題講解}}
  \begin{minted}[breaklines]{cpp}
    void push(Treap *t) { // 可以把 tag 想成要把底下的區間反轉幾次
      if (!t->tag) return;
      if (t->l) { // 把 tag 下推到左子樹
        swap(t->l->l, t->l->r);
        t->l->tag ^= t->tag;
      }
      if (t->r) {
        swap(t->r->l, t->r->r);
        t->r->tag ^= t->tag;
      }
      t->tag = false;
    }
  \end{minted}
\end{frame}

\begin{frame}[fragile]{\btitle{例題講解}}
  \begin{minted}[breaklines]{cpp}
    void reverse(int l, int r) {
      Treap *tl, *tr;
      split(root, l - 1, tl, root);
      split(root, r - l + 1, root, tr);
      swap(root->l, root->r); // 區間反轉,其實就是把左右子樹對調
      root->tag = true; // 紀錄 tag,代表往下的子樹也都需要左右對調
      root = merge(merge(tl, root), tr);
    }
  \end{minted}
\end{frame}

\begin{frame}[fragile]{\btitle{知道自己在哪}}
  \begin{itemize}
    \item 有了 size 之後,感覺就可以做很多壞事,比如說:
    \item 給定一個 Treap 節點,詢問這個節點在中序的位置(也就是這個節點的 size)
    \item 要怎麼做呢?
  \end{itemize}
\end{frame}

\begin{frame}[fragile]{\btitle{知道自己在哪}}
  \begin{minted}[breaklines]{cpp}
    struct Treap {
      Treap *lc, *rc, *fa;
      int sz, pri;
      Treap(): lc(NULL), rc(NULL), fa(NULL), sz(1), pri(rand()){};
    };

    int Size(Treap *t) {
      return t ? t->sz : 0;
    }
  \end{minted}
\end{frame}

\begin{frame}[fragile]{\btitle{知道自己在哪}}
  \begin{minted}[breaklines]{cpp}
    void pull(Treap *t) {
      t->fa = NULL;
      t->sz = 1 + Size(t->lc) + Size(t->rc);
      if (t->lc) {
        t->lc->fa = t;
        t->sz += t->lc->sz;
      }
      if (t->rc) {
        t->rc->fa = t;
        t->sz+= t->rc->sz;
      }
    }
  \end{minted}
\end{frame}

\begin{frame}[fragile]{\btitle{知道自己在哪}}
  \begin{minted}[breaklines]{cpp}
    int get_size(Treap *node) {
      int ret = Size(node->lc) + 1;
      while (node->fa != NULL) {
        if (node->fa->rc == node) {
          ret += 1 + Size(node->fa->lc);
        }
        node = node->fa;
      }
      return ret;
    }
  \end{minted}
\end{frame}

\begin{frame}[fragile]{\btitle{總結}}
  \begin{itemize}
    \item 一種利用隨機性質的平衡二元樹
    \item 可以解決線段樹無法處理的區間翻轉、...等等操作
    \item 操作都是建議在 merge、split 上面
    \item warning:Treap 的常數又比線段樹大了一些,如果遇到卡常題目請斟酌使用
  \end{itemize}
\end{frame}

\end{document}
